\documentclass[]{report}

\usepackage{amsthm}
\usepackage{biblatex}
\usepackage{csquotes}
\usepackage{amsmath,amssymb}
\usepackage{geometry}
\usepackage{tikz}
\usepackage{graphicx}
\usepackage{float}

\textwidth=450pt\oddsidemargin=0pt
\geometry{a4paper, top=3cm, bottom=3cm, left=3.4cm, right=3.4cm, % heightrounded, bindingoffset=5mm 
}
\theoremstyle{definition}
\newtheorem{Def}{Definizione}

\theoremstyle{Theorem}
\newtheorem{Theo}[Def]{Teorema}
\newtheorem{Prop}[Def]{Proposizione}

\newtheorem{Lm}[Def]{Lemma}

\theoremstyle{definition}
\newtheorem{Ex}[Def]{Esempio}

\theoremstyle{definition}
\newtheorem{Lem}[Def]{Lemma:}

\theoremstyle{definition}
\newtheorem{Obs}[Def]{Osservazione:}


%opening
\title{Equazioni di campo di Einstein e tensore energia momento}
\author{Gabriele Novelli}
\date{}

\begin{document}

\maketitle

\section{Introduzione}
In questa tesina verranno trattate a livello introduttivo le equazioni di campo di Einstein ed il tensore energia momento.\\
\\
Nella prima sezione si tratterà solamente del tensore energia momento e delle sue proprietà. Verranno fornite alcune dimostrazioni per quanto riguarda la simmetria di questo e la legge di conservazione dell'energia. Verrà infine analizzato il caso di fluidi perfetti. Questo caso è particolarmente importante per la teoria della relatività generale in quanto generalmente si suppone che l'universo sia composto da un fluido perfetto di materia ed energia.\\
\\
Nel secondo capitolo invece verranno introdotte le equazioni di Einstein. Si analizzerà innanzitutto il comportamento del tensore metrico in approssimazione di campo debole. Successivamente si fornirà una dimostrazione del tensore di Einstein, a partire dalle proprietà note sul tensore energia momento e sulla metrica. Nell'ultima parte di questa tesina invece si studierà la linearizzazione del tensore di Einstein.
\newpage
\section{Tensore energia momento}
\subsection{Definizione}
L'introduzione del tensore energia momento è dovuta alla necessità di poter esprimere in modo sintetico il flusso di energia attraverso una determinata superficie.
Per una generica particella, le grandezze energia $E$ e momento $p$ sono contenute dentro al quadri-impulso, indicato con $p^\mu=mu^\mu=m\gamma(c,u_x,u_y,u_z)$. Una volta noto questo vettore, per estrarne l'energia è necessaria una 1-forma che selezioni la componente 0. Allo stesso modo, per definire una densità di flusso attraverso una superficie a t costante, è necessaria una 1-forma. Si definisce allora il tensore energia momento come il tensore \textbf{T} le cui componenti
$$T(\tilde{dx^\alpha},\tilde{dx^\beta})=T^{\alpha\beta}$$
rappresentano il flusso di momento $p^\alpha$ attraverso la superficie ad $x^\beta$ costante.\\
Analizziamo con più attenzione le componenti:
$T^{00}$ è il flusso di $p^0$, ovvero dell'energia, attraverso le superfici ad $x^0=t$ costante. Quindi, per definizione, $T^{00}$ è la densità di energia.
Allo stesso modo, $T^{0i}$ è il flusso di energia attraverso le superfici a $x^i$ costante.
$T^{i0}$ è il flusso della componente $p^i$ del momento attraverso le superfici a $t$ costante, mentre $T^{ij}$ è il flusso della i-esima componente del momento $p^i$ attraverso le superfici a $x^i$ costante.
Cerchiamo ora di dare una definizione più esplicita delle componenti in un sistema di riferimento (SDR) in co-movimento con la particella, indicato con $S$.
\subsection{Tensore energia momento nella polvere}
Se volgiamo calcolare le componenti di \textbf{T} per la polvere, nel SDR scelto non c'è movimento delle particelle, ovvero tutte le componenti del momento sono nulle ed i flussi sono nulli. Quindi l'unica componente non nulla del tensore energia momento è $T^{00}$. Allora, se $n$ è il numero di particelle per unità di volume, si ha $T^{00}=mn\gamma=\rho$ (in questo caso $\rho$ è la densità di energia e abbiamo posto $c=1$). \\
Cioè è lecito scrivere
$$\textbf{T}=mn\vec{U}\otimes\vec{U}=\rho\vec{U}\otimes\vec{U}$$
Dove $U^\mu$ è la quadri-velocità del fluido.
Da questa analisi possiamo concludere che, spostando il sistema di riferimento in un nuovo $SDR$ indicato con $S'$, si possono calcolare le componenti del tensore \textbf{T} nel seguente modo:
$$T^{\alpha\beta}=\textbf{T}(\tilde{dx^\alpha},\tilde{dx^\beta})=\rho U^\alpha U^\beta$$
Dato che in questo generico sistema di riferimento le componenti del quadrivettore $U^\mu$ sono date da $U^\mu=({1\over\sqrt{1-v^2}},{u_x\over\sqrt{1-v^2}},{u_y\over\sqrt{1-v^2}},{u_z\over\sqrt{1-v^2}})$, ne segue che la generica componente del tensore $\textbf{T}$ ha l'espressione:
$$T^{ij}={\rho v^iv^j\over \sqrt{1-v^2}}$$

\begin{Obs}
	Notiamo dall'ultima relazione che il tensore energia momento $\textbf{T}$ è, per la polvere, simmetrico. In realtà la simmetria è un proprietà intrinseca di questo tensore ed è vera in generale. Ricordiamo inoltre un importantissimo risultato noto dall'algebra lineare: una qualsiasi matrice simmetrica può essere sempre messa in forma diagonale. In altre parole, se $\textbf{T}$ è simmetrico allora esiste sempre un sistema di riferimento in cui esso assume la forma diagonale.
\end{Obs}
\subsection{Simmetria e conservazione dell'energia}
In questa sezione proveremo che il tensore energia momento è simmetrico e lo utilizzeremo per ricavare la legge di conservazione dell'energia. Supporremo di analizzare un generico fluido e utilizzeremo come sistema di riferimento quello co-movente con l'elemento di fluido infinitesimo scelto.\\ \\
Proviamo ora che il tensore energia momento è effettivamente simmetrico e che quindi può essere diagonalizzato.
\begin{Theo}
	Il tensore $\textbf{T}$ è simmetrico.
\end{Theo}
\begin{proof}
	Ci basta dimostrare la simmetria di $\textbf{T}$ in un solo sistema di riferimento. Mettiamoci dunque nel SDR co-movente con l'elemento di fluido, indicato con S. Proviamo innanzitutto che la parte di tensore individuata dalle coordinate $i,j>0$ è simmetrica.
		\begin{figure} [H]
			\centering
			\label{Image 1}
			\includegraphics{TSimmetric.JPG}
	\caption{Elemento di fluido nel SDR comovente.}	
	\end{figure}
	Si consideri un elemento di fluido cubico di lato $l$ come in figura \ref{Image 1}. Allora la forza che questo esercita sulla faccia j-esima dell'elemento di fluido adiacente è data da $F^i_j=T^{ij}l^2$ con $i$ che varia da $1$ a 3. Una forza è applicata anche sulla faccia opposta. Indichiamo questa seconda forza con $-F^i_j$ in quanto, se portiamo $l$ a 0, le superfici vanno a coincidere.\\
	Lo stesso ragionamento si può applicare anche alle altre facce. \\
	Calcoliamo ora i momenti torcenti generati da queste forza rispetto all'asse z (ovviamente, non consideriamo i contributi in direzione z). Per questo calcolo, spostiamo il centro del SDR nel centro del cubo di fluido. Inoltre operiamo una piccola approssimazione: consideriamo le forze come agenti sul centro delle facce considerate. Indichiamo poi con $r$ la distanza tra l'asse $z$ e il punto di applicazione della forza.\\
	Il momento torcente dovuto a $\vec{F_1}$ è $(\vec{r}\times\vec{F_1})^z=xF_x^y$. Il contributo della forza sulla faccia opposta è lo stesso, per simmetria.
	Se sostituiamo $x=l/2$ e $F^i_j=T^{ij}l^2$ otteniamo:
	$(\vec{r}\times\vec{F_1})^z={l^3\over 2}T^{yx}$. Possiamo applicare lo stesso ragionamento alle altre due facce, ottenendo come risultato $(\vec{r}\times\vec{F_2})=-{l^3\over 2}T^{xy}$. Quindi il momento torcente totale è dato da $$\tau^z=l^3(T^{xy}-T^{yx})$$
	Osserviamo poi che il momento d'inerzia dell'elemento cubico di volume rispetto al suo asse z è proporzionale a $I\propto ml^2=\rho l^5$. Ma allora l'accelerazione angolare ha la forma $$\ddot{\theta}={\tau\over I}={T^{xy}-T^{yx}\over \rho l^2}$$
	Per $l\rightarrow 0$ questa quantità tende ad infinito a meno che $T^{xy}=T^{yx}$. Dato che sperimentalmente non osserviamo il fluido in movimenti vorticosi, sarà necessario che $T^{xy}=T^{yx}$. Questo termina la prima parte della dimostrazione.\\
	\\
	Per dimostrare la simmetria delle rimanenti componenti del tensore invece è richiesto meno sforzo. Ricordiamo che $T^{0i}$ è il flusso di energia attraverso una determinata superficie ad $x^i$ costante. Dato che l'energia corrisponde alla massa, in realtà $T^{0i}$ è la densità di massa moltiplicata per la velocità a cui si sta muovendo, ovvero la densità di momento. Quindi $T^{0i}=T^{i0}$.
	Questo termina la dimostrazione.
\end{proof}

La costruzione del tensore $\textbf{T}$ presuppone che esista un modo per ricavare, a partire da esso, la legge di conservazione dell'energia. Ciò è effettivamente possibile e la relazione che si trova è:
\begin{Theo} \label{Teo 2} La legge di conservazione dell'energia può essere espressa mediante il tensore energia momento come:
	$${\partial \over \partial x^i}T^{0i}=0$$
	Questa relazione inoltre vale anche per tutti gli altri indici.	
\end{Theo}
\begin{Obs}
Prima di partire con la dimostrazione è giusto sottolineare che, per il principio di covarianza generale, le leggi della relatività generale si ottengono a partire da quelle della relatività speciale sostituendo le quantità tensoriali del gruppo di Lorentz con quelle della varietà spazio tempo. In questa ottica ${\partial \over \partial x^i}$ è rimpiazzato da $\nabla_i$ derivazione covariante. Quindi il teorema \ref{Teo 2} in realtà andrebbe riscritto come:
$$\nabla_\mu T^{\nu\mu}=0$$
Notiamo che questa riscrittura sostanzialmente comporta l'inserimento di un nuovo termine dipendente dai simboli di Christoffel.
\end{Obs}
\begin{proof}
	Si consideri un elemento infinitesimo di fluido cubico di lato $l$.
	\begin{figure} [H]
		\centering
		\label{Image 2}
		\includegraphics{TCons.JPG}
		\caption{Sezione dell'elemento di fluido nel SDR comovente, a z costante.}	
	\end{figure}
Si faccia riferimento alla figura \ref{Image 2}. Il flusso di energia attraverso la faccia 4 è dato da $l^2T^{0x}(x=0)$ ed attraverso la faccia 2 da $-l^2T^{0x}(x=l)$. Similmente, per le facce 1 si avrà $l^2T^{0y}(y=0)$ e 3 $-l^2T^{0y}(y=l)$.\\
Per conservazione dell'energia, la somma di questi contributi dovrà equivalere l'effettivo cambiamento di energia interno al volume, ovvero:
$${\partial \over \partial t}(T^{00}l^3)=l^2[T^{0x}(x=0)-T^{0x}(x=l)+T^{0y}(y=0)-T^{0y}(y=l)+T^{0z}(z=0)-T^{0z}(z=l)]$$
Dividiamo ora per $l^3$ ed operiamo il limite per $l\rightarrow 0$, ottenendo:
$${\partial \over \partial t}T^{00}=-{\partial \over \partial x}T^{0x}-{\partial \over \partial y}T^{0y}-{\partial \over \partial z}T^{0z}$$
Per le altre componenti del tensore energia momento il ragionamento è analogo.
Questo completa la dimostrazione.
\end{proof}
\subsection{Fluidi perfetti}
In questa sezione analizziamo il tensore energia momento in un fluido perfetto.\\
\\
In relatività generale si dice fluido perfetto un fluido che, nel SDR di riferimento co-movente con l'elemento analizzato, non presenta nè viscosità nè conduzione di calore.
\begin{Obs}
	In un fluido perfetto non c'è conduzione di calore. Questo implica che $T^{0i}=T^{i0}=0$. L'energia può fluire solamente se fluisce la materia.\\
	La viscosità di un fluido è il risultato delle forze di interazione parallele alle superfici tra elementi di fluido adiacenti. L'assenza di viscosità implica l'assenza di queste forze che quindi dovranno risultare necessariamente tutte perpendicolari alle superfici in questione. Da ciò, $T^{ij}=0$ a meno che $i=j$.
	Inoltre, l'assenza di viscosità deve valere in tutti i sistemi di riferimento co-moventi con gli elementi di fluido. Quindi è necessario che $\textbf{T}$ (escludendo la parte energetica) sia diagonale in tutti questi sistemi. L'unica matrice che soddisfa questa richiesta è l'identità, a meno di una costante. E' allora lecito supporre che $\textbf{T}$ abbia, in un fluido perfetto, la forma:
	$$\textbf{T}=\begin{pmatrix}
		\rho & 0 & 0 &0\\
		0 & p & 0 &0\\
		0 & 0 & p &0\\
		0 & 0 & 0 &p\\
	\end{pmatrix}$$	
\end{Obs}
\begin{Obs}
	Non è difficile mostrare che il tensore energia momento si può riscrivere anche nel seguente modo:
	$$T^{ij}=(p+\rho)U^iU^j+p\eta^{ij}$$
	dove con $\eta^{ij}$ si è indicata la metrica di Minkowski e con $U^i$ la quadri-velocità dell'elemento di fluido considerato. Applicando nuovamente il principio di covarianza generale, otteniamo
	$$T^{ij}=(p+\rho)U^iU^j+pg^{ij}$$
	dove alla metrica di Minkowski si è sostituito un generico tensore metrico $g^{ij}$ e $U^\mu$ è la quadri-velocità dell'elemento di fludio calcolata nel SDR co-movente con esso.
\end{Obs}
\section{Equazioni di campo di Einstein}
In questa sezione ricaviamo le equazioni di campo di Einstein e ne studiamo il comportamento in approssimazione di campo debole. Per la buona riuscita di questa trattazione è anche necessario determinare una corretta approssimazione per la metrica nel caso di campo Newtoniano.
\subsection{Metrica in approssimazione di campo debole}
In questa sotto-sezione analizziamo la metrica in approssimazione di campo Newtoniano e cerchiamo di ricavare un'espressione che in seguito ci permetta di ricavare le equazioni di Einstein.\\
\\
Consideriamo dunque una particella in moto in un campo debole con velocità non relativistica. Dato che questa non è soggetta ad alcuna forza, eccetto che per la gravità, essa si muoverà lungo una geodesica. L'equazione che ne descrive il moto è:
$${d^2x^\alpha\over d\tau^2}+\Gamma^\alpha_{ij}{dx^i\over d\tau}{dx^j\over d\tau}=0$$ 
Se la particella è abbastanza lenta, consideriamo solo le componenti temporali:
$${d^2x^\alpha\over d\tau^2}+\Gamma^\alpha_{00}\bigg{(}{dt\over d\tau}\bigg{)}^2=0$$
E' inoltre noto che i somboli di Christoffel hanno la forma
$$\Gamma_{ij}^\alpha={1\over 2}g^{\alpha\beta}\bigg{(}{\partial g_{i\beta}\over \partial x^j}+{\partial g_{j\beta}\over \partial x^i}-{\partial g_{ij}\over \partial x^\beta}\bigg{)}$$
Dato che il campo è statico, tutte le derivate temporali di $g_{ij}$ svaniscono, da cui:
$$\Gamma^\alpha_{00}=-{1\over 2}g^{\alpha\beta}{\partial g_{00}\over \partial x^\beta}$$
Inoltre, poiché il campo è debole, è lecito supporre che la metrica $g_{ij}$ sia vicina a quella di Minkowsi. Questo ci porta a scrivere
$$g_{ij}=\eta_{ij}+h_{ij}$$
Con la condizione che $|h_{ij}|<<1$. Data la piccolezza del termine aggiuntivo $h_{ij}$, siamo sicuri che la metrica sia non troppo distante da quella Minkowskiana. Inseriamo questa ultima deduzione all'interno dell'espressione per $\Gamma$, ricordando che i $\eta_{ij}$ è costante (quindi avrà derivata nulla) e troviamo:
$$\Gamma^\alpha_{00}=-{1\over 2}\eta^{\alpha\beta}{\partial h_{00}\over \partial x^\beta}$$
Riprendiamo poi l'equazione della geodesica e sostituiamoci dentro questo ultimo risultato:
$${d^2x^\alpha\over d\tau^2}={1\over 2}\bigg{(}{dt\over d\tau}\bigg{)}^2\nabla h_{00}$$
Supponendo $dt\over d\tau$ costante e dividendo per questo termine al quadrato, abbiamo:
$${d^2x^\alpha\over dt^2}={1\over 2}\nabla h_{00}$$
Possiamo quindi identificare la quantità ${1\over 2}h_{00}$ come un potenziale. Il corrispondente risultato newtoniano è:
$$\nabla\phi=-{d^2x^\alpha\over dt^2}$$
dove $\phi=-{GM\over r}$.
Questa analogia ci suggerisce di porre $h_{00}=-2\phi+c$ dove $c$ è una costante da determinare. Tuttavia il sistema di coordinate deve divenire Minkowskiano all'infinità spaziale, ovvero: $h_{00}\rightarrow 0$ a grandi distanze. A questa considerazione aggiungiamo che anche $\phi$ deve svanire a grandi distanze. Da ciò: $c=0$ e quindi: $h_{00}=-2\phi$.\\
\\
In conclusione, nel caso di un campo debole abbiamo ricavato la relazione:
$$g_{00}=-1-2\phi$$
\subsection{Derivazione delle equazioni di Einstein}
In questa sezione si enuncerà una pseudo-dimostrazione delle equazioni di campo di Einstein, a partire dai risultati trovati nei capitoli precedenti e da alcuni altri assunti noti.
Ricordiamo che in un qualsiasi punto $p$ della varietà spazio-tempo possiamo sempre esprimere la metrica in forma diagonale e canonica:
$$g^{ij}=\eta^{ij}$$
Inoltre è noto che in un intorno $U$ abbastanza piccolo di $p$, la metrica differirà da $\eta^{ij}$ solamente per termini quadratici in $\delta x$. Assumiamo che in $U$ il campo sia debole. L'approccio dimostrativo sarà quello di cercare di scrivere le equazioni in questa  approssimazione, per poi ri-scriverle in modo più generale tramite alcune osservazioni. Ricordiamo che la componente $g_{00}$ della metrica, immersa in un campo debole prodotto da una densità di massa non relativistica $\rho$, è data da $g_{00}=-1-2\phi$; dove $\phi$ è determinato dall'equazione $$\nabla^2\phi=4\pi G\rho$$
Ricordiamo poi che $T_{00}=\rho$. Combinando questi risultati otteniamo:
\begin{equation}\label{Equation 1}
	\nabla^2g_{00}=8\pi GT_{00}
\end{equation}

La struttura di questa equazione rende lecito ipotizzare che l'equazione di campo di Einstein abbia la forma 
$$G_{ij}=8\pi GT_{ij}$$
dove $G_{ij}$ si suppone essere una combinazione lineare della metrica e delle sue derivate prime e seconde.

\begin{Obs}
	Dall'equazione \ref{Equation 1} è possibile fare alcune deduzioni. Prima di tutto, data la simmetria del tensore $\textbf{T}$, anche $G_{ij}$ sarà necessariamente simmetrico. Inoltre, l'unità di misura di $G_{ij}$ deve essere la stessa della derivata seconda del tensore metrico. Quindi ogni termine di grado $N$ dovrà essere moltiplicato per una costante con le dimensioni di una lunghezza elevata alla $N-2$. Ciò rende i termini di grado successivo al secondo trascurabili per scale di tempi ridotte e per lunghezze molto estese, nel caso $N\neq 2$.
	Per rimuovere questa ambiguità, si suppone che il tensore $G_{ij}$ contenga solamente termini alle derivate seconde della metrica. Ovviamente l'espressione di $G_{ij}$ che si calcolerà dovrà essere tale da ridursi all'equazione \ref{Equation 1} nel limite di campo debole.
	Infine, data la conservazione del tensore energia momento, espressa dalla legge 
	$$\nabla_\mu T^{\mu\nu}=0$$
	dovrà valere anche 
	$$\nabla_\mu G^{\mu\nu}=0$$
\end{Obs}
Le osservazioni fatte sono sufficienti per trovare un'espressione di $G_{ij}$.\\
\\
E' possibile dimostrare che il tensore $R_{ijkl}$ di Riemann è l'unico tensore che si può ottenere utilizzando il tensore metrico e le sue derivate prime e seconde. Data la proprietà di antisimmetria di questo tensore, contraendolo possiamo formare solamente i tensori di Ricci $R_{ij}=R^k_{ikj}$ e di curvatura $R=R^\mu_\mu$. Riprendendo le osservazioni precedenti, possiamo scrivere $G_{ij}$ nella forma generale:
$$G_{ij}=C_1g_{ij}R+C_2R_{ij}$$
dove $C_1$ e $C_2$ sono costanti arbitrarie da determinare.
Osserviamo che $G_{ij}$ così definito è automaticamente simmetrico.\\
\\
Il prossimo passaggio è applicare la condizione che abbiamo ottenuto dalla conservazione dell'energia: $$\nabla_\mu G^{\mu\nu}=0$$
Dall'identità di Bianchi è noto che:
$$\nabla_i({1\over 2}g^{ij}R-R^{ij})=0$$
Questo suggerisce di porre $C1=-C2/2$, ottenendo come forma generale per il tensore di Einstein:
\begin{equation}
	\label{Equation 2}
	G_{ij}=C_2(R_{ij}-{1\over 2}g_{ij}R)
\end{equation}
Rimane ora il problema della determinazione della costante $C_2$. Per risolvere questo ultimo passaggio, sfruttiamo la condizione per cui, nel limite di campo debole, si deve ritrovare l'equazione \ref{Equation 1}.\\
Supponendo un campo debole, la metrica sarà $g_{ij}\simeq \eta_{ij}$, ovvero differirà di poco da quella di Minkowski. Allora l'espressione della traccia del tensore $R_{ij}$ sarà ($k\neq 0$):
$$R=R_{kk}-R_{00}$$
Inoltre, dal fatto che un sistema relativistico ha sempre $|T_{ij}|<<|T_{00}|$, dovrà anche valere $|G_{ij}|<<|G_{00}|$ (per $i,j\neq 0$) che, riprendendo l'espressione \ref{Equation 2} si traduce in
$$R_{ij}\simeq{1\over 2}g_{ij}R$$
Allora la traccia del tensore $R_{ij}$ sarà data da:
$$R=R_{kk}-R_{00}={3\over 2}R-R_{00}$$
Espressione che si traduce in $2R_{00}=R$. Sostituendo questa relazione dentro all'equazione \ref{Equation 2} troviamo:
\begin{equation}
	\label{Equation 3}
	G_{00}=2C_2R_{00}
\end{equation} 
Per calcolare questo contributo possiamo utilizzare l'espressione generale di $R_{ijkl}$, considerando la parte lineare nelle derivate seconde della metrica. In particolare:
$$R_{ijkl}={1\over 2}\bigg{[}{\partial^2 g_{ik} \over\partial x^j\partial x^l}-{\partial^2 g_{jk}\over\partial x^i\partial x^l}-{\partial^2 g_{il} \over\partial x^j\partial x^k}+{\partial^2 g_{jl}\over\partial x^i\partial x^k}\bigg{]}$$
Dato che il campo è statico, tutte le derivate temporali sono nulle e le componenti che ci servono sono solamente
\begin{align*}
	R_{0000}&\simeq 0 & R_{i0j0}&\simeq {1\over 2} {\partial^2 g_{00} \over\partial x^j\partial x^i}  
\end{align*}
Da queste determiniamo l'espressione ricercata per $R_{00}$, che ci riconduce proprio all'espressione dell'equazione \ref{Equation 1}.
$$R_{00}=R_{i0j0}-R_{0000}\simeq {1\over 2} {\partial^2 g_{00} \over\partial x^j\partial x^i}\simeq {1\over 2} \nabla^2 g_{00}$$
Combinando questo ultimo risultato con quello trovato nell'equazione \ref{Equation 3} si ottiene finalmente:
$$G_{00}=C_2\nabla^2g_{00}$$
Confrontando infine questa formula con l'equazione \ref{Equation 1}, è evidente che la soluzione si ha per $C_2=1$. In conclusione, le equazioni di campo di Einstein sono date da:
$$G_{ij}=R_{ij}-{1\over 2}g_{ij}R=8\pi GT_{ij}$$
\begin{Obs}
	Possiamo contrarre il tensore di Einstein con $g^{ij}$ per ottenere una quantità scalare proporzionale alla curvatura scalare (la traccia del tensore di Ricci):
	$$g^{ij}(R_{ij}-{1\over 2}g_{ij}R)=g^{ij}(8\pi GT_{ij})$$
	$$R-2R=-R=8\pi GT$$
	Se poi ci troviamo nel vuoto, $\textbf{T}$ svanisce e quindi ci ritroviamo con l'equazione $R_{ij}=0$.
\end{Obs}
\subsection{Il limite Newtoniano}
In questa sezione cerchiamo ora un'espressione per il tensore di Einstein nel limite Newtoniano di campo debole. Per comodità, utilizzeremo a volte la seguente notazione per indicare l'operazione di derivazione:
$${\partial\over\partial x^i}h^{k}=h^k_{,i}$$
\\
Come fatto in precedenza, assumiamo che la metrica non sia distante da quella di Minkowski, ovvero che valga l'ipotesi:
$$g_{ij}=\eta_{ij}+h_{ij}$$
dove $\eta_{ij}$ è la metrica di Minkowski e $h_{ij}$ è una correzione tale che $|h_{ij}|<<1$. Prima di procedere con la linearizzazione del tensore di Einstein, è opportuno classificare un'importante classe di trasformazioni.\\
\\
Consideriamo un cambio di coordinate infinitesimo $x^{i'}=x^i+\xi^i(x^j)$ dove $\xi^i$ è una funzione della posizione, molto piccola: $|\xi^i|<<1$. Supponiamo anche che le derivate prime di $\xi^i$ siano molto piccole: $|{\partial\xi^i\over \partial x^m}|<<1$. E' possibile dimostrare che per una trasformazione di questo tipo vale:
$$h_{\alpha\beta}\rightarrow h_{\alpha\beta}-{\partial\xi_{\alpha}\over \partial x^\beta}-{\partial\xi_{\beta}\over \partial x^\alpha}=
h_{\alpha\beta}-\xi_{\alpha,\beta}-\xi_{\beta,\alpha}$$
Questo tipo di trasformazione è detta di Gauge. 
Utilizzeremo questa trasformazione per ridurre il tensore di Riemann in una forma più sintetica. Prima di procedere con il calcolo, ci si posizionerà in un punto in modo da costruire un sistema di riferimento normale. In questo modo i coefficienti di Christoffel saranno nulli ed il tensore di Riemann avrà la forma:
$$R_{jikl}={1\over 2}\bigg{[}g_{ik,jl} -g_{jk,il}-g_{il,jk} + g_{jl,ik}\bigg{]}$$
Inoltre, prima di utilizzare la relazione $h_{\alpha\beta}\rightarrow h_{\alpha\beta}-\xi_{\alpha,\beta}-\xi_{\beta,\alpha}$, facciamo la seguente osservazione: dato che la nostra metrica è composta da due termini, di cui uno costante, (ovvero $\eta_{ij}$) per linearità dell'operazione di derivazione si scrive:
$$R_{jikl}={1\over 2}\bigg{[}h_{ik,jl} -h_{jk,il}-h_{il,jk} + h_{jl,ik}\bigg{]}$$
L'idea è ora quella di contrarre il tensore di Riemann in modo da ottenere il tensore di Ricci $R_{ij}$ e lo scalare $R$, così da costruire $G_{ij}=R_{ij}-{1\over2}Rg_{ij}$ tensore di Einstein. Prima di enunciare l'espressione del tensore, chiamiamo $\bar{h}_{ij}=h_{ij}-{1\over 2}h$ che prende il nome di tensore "traccia inversa di h". Infatti, $\bar{h}=-h$.
Si può mostrare che, contraendo il tensore di Riemann e sostituendo il tensore traccia inversa, il tensore di Einstein ha la forma:
\begin{equation}
	\label{Equation 3}
	G_{ij}=-{1\over 2}\bigg[\bar{h}_{ij,\mu}^{\hspace{8pt},\mu}+\eta_{ij}\bar{h}_{\mu\nu}^{\hspace{8pt},\mu\nu}-\bar{h}_{i\mu,j}^{\hspace{8pt},\mu}-\bar{h}_{j\mu,i}^{\hspace{8pt},\mu}\bigg]+O((h_{ij})^2)
\end{equation}

\begin{Obs}
	Notiamo che l'espressione \ref{Equation 3} si semplificherebbe se fosse verificato che $\bar{h}^{\mu\nu}_{,\nu}=0$. Infatti, in questa situazione l'equazione \ref{Equation 3} si ridurrebbe a:
	$$	G_{ij}=-{1\over 2}\bigg[\bar{h}_{ij,\mu}^{\hspace{8pt},\mu}\bigg]+O((h_{ij})^2)$$ 
	Cerchiamo allora una trasformazione del tipo $h_{\alpha\beta}\rightarrow h_{\alpha\beta}-\xi_{\alpha,\beta}-\xi_{\beta,\alpha}$ che renda vera questa richiesta.		
\end{Obs}
Partiamo con il trovare una trasformazione di questo tipo: sia $h_{ij}^{new}=h_{ij}^{old}-\xi_{i,j}-\xi_{j,i}$. Allora, per definizione di tensore traccia inversa:
$$\bar{h}_{ij}^{new}=h_{ij}^{new}-{1\over 2}\eta_{ij}h^{new}=h_{ij}^{old}-\xi_{i,j}-\xi_{j,i}-{1\over 2}\eta_{ij}(h^{old}-\xi_{i,j}-\xi_{j,i}-\xi_{\alpha,\alpha})$$
$$=h_{ij}^{old}-{1\over 2}\eta_{ij}h^{old}-\xi_{i,j}-\xi_{j,i}+\eta_{ij}\xi_{,\alpha}^\alpha=$$
$$=\bar{h}_{ij}^{old}-\xi_{i,j}-\xi_{j,i}+\eta_{ij}\xi_{,\alpha}^\alpha$$
Ora deriviamo ed imponiamo che $\bar{h}_{\mu\nu}^{,\nu \, (new)}=0$:
$$\bar{h}_{\mu\nu}^{,\nu \, (new)}=0=\bar{h}_{\mu\nu}^{,\nu \, (old)}-\xi_{\mu,\nu}^{\hspace{8pt},\nu}-\xi_{\nu,\mu}^{\hspace{8pt},\nu}+(\eta_{\mu \nu}\xi_{,\alpha}^\alpha)^{,\nu}$$
Si suppone ora che i termini $\xi$ derivati due volte in modo misto siano trascurabili. Dato che la derivata del tensore metrico è nulla (in quanto la metrica di Minkowski è costante), ciò che rimane è:
$$\bar{h}_{\mu\nu}^{,\nu \, (new)}=0=\bar{h}_{\mu\nu}^{,\nu \, (old)}-\xi_{\mu,\nu}^{\hspace{8pt},\nu}$$
Utilizzando come notazione il simbolo di D'Alembert $\square\xi_\mu=\xi_{\mu,\nu}^{,\nu}$ possiamo riscirvere il tensore di Einstein a partire dall'equazione \ref{Equation 3} trascurando il resto:
$$G_{ij}=-{1\over 2}\square \bar{h}_{ij}^{new}=8\pi G T_{ij}$$
\begin{Obs}
	Osserviamo che non vi è unicità nelle trasformazioni di Gauge. Infatti una volta individuato $\xi_{mu}$ tale che $0=\bar{h}_{\mu\nu}^{,\nu \, (old)}-\xi_{\mu,\nu}^{\hspace{8pt},\nu}$, possiamo sempre costruire $\xi_\mu+A_\mu$ con $\square A_{\mu}=0$ in modo che $\bar{h}_{\mu\nu}^{,\nu \, (old)}=\square(\xi_\mu+A_\mu)$.
\end{Obs}

\chapter*{Bibliografia}
\begin{itemize}
	\item[$\circ$] [1] Steven Weinberg ,Gravitation And Cosmology: Principles And Applications Of The General Theory Of Relativity, Wiley, 1972; 
	\item[$\circ$] [2] Bernard Schutz, A First Course in General Relativity, Cambridge University Press, 2009
\end{itemize}

\end{document}
