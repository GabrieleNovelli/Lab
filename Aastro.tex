\documentclass{beamer}
\usetheme{Ilmenau}
\usepackage[english]{babel}

\title{Inferring the falvor of high energy astrophysical neutrinos at their sources}
\begin{document}
\begin{frame}
    \frametitle{Introduction}
    \begin{itemize}
    	\item What?
    	\\
    	A method to infer the flavor of high energy astrophysical neutrinos at their sources (based on the measurments at Earth).
    	\item Why?\\
    	Since sources of HEN are still unknown, this may help to determine the physical properties of them.
    	\item How?\\
    	We need to account for the different reactions that produce neutrinos, for the oscillations and for the uncertanties in the mixing parameters.
    \end{itemize}
\end{frame}
\frametitle{Production}
\begin{alertblock}{High energy Pion decay}
	\begin{center}
		$\pi^+\rightarrow \mu^++\nu_\mu$	\\
		$\mu^+\rightarrow\bar{\nu_\mu}+\nu_e+e^+$\\
		expected $(1_e:2_\mu:0_\tau)_S$. (Assume $\nu$ and $\bar\nu$ contrinbute equally).
	\end{center}
\end{alertblock}
There are other production reactions:
\begin{block}{$\mu$ dumped}
	$\mu$ loses energy due to synchrotron radiation and thus lower energy neutrinos are produced. For high energy neutrinos we expect $(0_e:1_\mu:0_\tau)_S$.
\end{block}
\begin{block}{$n$ decay}
	$n$ co-produced with pions $\beta-decay$ into $\bar\nu_e$. (Neutrinos produced are 100 times less energetic).
\end{block}
\begin{frame}
	\frametitle{Neutrino Oscillations}
	$$f_{\alpha,\oplus}=\sum_\beta P_{\beta\alpha}f_{\beta,S}$$ are the expected fluxes at Earth.
	$$P_{\alpha\beta}=\sum_{i}|U_{\alpha i}|^2|U_{\beta i}|^2$$ probabilities of oscillation.
\end{frame}
\begin{frame}
	\frametitle{Measuring flavours at Earth}
	Measures are done through IceCube. But how?
	\begin{itemize}
		\item Topology of the tracks
		
	\end{itemize}
\end{frame}
\end{document}
