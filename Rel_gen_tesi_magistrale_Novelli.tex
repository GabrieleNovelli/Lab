\documentclass[]{article}

\usepackage[english]{babel}
\usepackage{newlfont}
\usepackage{color}
\usepackage{float}
\usepackage{frontespizio}
\usepackage{amsmath,amssymb}
\usepackage{amsthm}
\usepackage{geometry}
\usepackage{tikz}
\usepackage{biblatex}
\usepackage{csquotes}
\usepackage{pgfplots}
\usepackage{hyperref}
\usepackage{amssymb}
\usepackage{comment}
\usepackage{graphicx}

\hypersetup{
	colorlinks=true,
	linkcolor=blue,
	filecolor=magenta,      
	urlcolor=cyan,
	pdftitle={Overleaf Example},
	pdfpagemode=FullScreen,
}

\textwidth=450pt\oddsidemargin=0pt
\geometry{a4paper, top=3cm, bottom=3cm, left=3cm, right=3cm, % heightrounded, bindingoffset=5mm 
}
\theoremstyle{definition}
\newtheorem{Def}{Definizione}

\theoremstyle{Theorem}
\newtheorem{Theo}[Def]{Teorema}
\newtheorem{Prop}[Def]{Proposizione}

\newtheorem{Lm}[Def]{Lemma}

\theoremstyle{definition}
\newtheorem{Ex}[Def]{Esempio}

\theoremstyle{definition}
\newtheorem{Lem}[Def]{Lemma:}

\theoremstyle{definition}
\newtheorem{Obs}[Def]{Osservazione:}


%opening
\title{Schwarzschild Black Holes}
\author{Novelli Gabriele}
\date{ }

\begin{document}

\maketitle

\section{The Schwarzschild Metric}
In this section we will determine the Schwarzschild metric, which will be the starting point for our analysis on black holes.\\
Consider a spherically simmetric source in the vacuum. Our aim in this section is to determine the metric outside of this source. To carry on the calulations, we need to solve Einstein equation, which in the vacuum takes the simple form:
$$R_{\mu\nu}=0$$
We also assume that the source is not evolving in time. This, toghether with the spherical symmetry, translates in the following requirements: 
\begin{itemize}
	\item static source $\rightarrow$ all of the metric components are independent of time and we have no space-time cross terms;
	\item spherical symmetry $\rightarrow$ the metric takes the following general form:
	$$ds^2=-e^{2\alpha(r)}dt^2+e^{2\beta(r)}dr^2+e^{2\gamma(r)}r^2d\Omega^2$$
\end{itemize}
The exponentials have been choosen so that the signature of the metric does not change. We now define the new coordinate $\bar{r}=e^{\gamma(r)}r$ which implies obviously that $$d\bar{r}=e^{\gamma(r)}[1+r{d\gamma\over dr}]dr$$
Substituting this coordinate inside of the metric we get the expression:
$$ds^2=-e^{2\alpha(\bar{r})}dt^2+e^{2\beta(\bar{r})-2\gamma(\bar{r})}\bigg(1+r{d\gamma\over dr}\bigg)^{-2}d\bar{r}^2+\bar{r}^2d\Omega^2$$
Where the functions of $r$ have now become functions of $\bar{r}$, defined implicitly through the previous relation. Our next step is to define another variable:
$$e^{2\bar{\beta}(\bar{r})}=e^{2\beta(\bar{r})-2\gamma(\bar{r})}\bigg(1+r{d\gamma\over dr}\bigg)^{-2}$$ 
As for now we are just simply re-labelling our coordinates. Later we will worry about the physical interpretation of the coordinates we are using. With this substitution the metric takes the form:
$$ds^2=-e^{2\alpha(\bar{r})}dt^2+e^{2\bar\beta(\bar{r})}d\bar{r}^2+\bar{r}^2d\Omega^2$$
Now we can determine the general form of the Riemann tensor. In ordedr to achieve this we compute the Christoffel Symbols. Recall that for a metric connection $\Gamma_{\mu\nu}^\beta=\Gamma_{\nu\mu}^\beta$ we have the relation:
$$\Gamma_{\mu\nu}^\beta={1\over 2}g^{\beta\alpha}(g_{\mu\alpha,\nu}+g_{\nu\alpha,\mu}-g_{\mu\nu,\alpha})$$
Using as indices ${t,r,\sigma,\phi}$ (to simplify the notation we have renamed $\bar{r}$ to $r$ and $\bar{\beta}$ to $\beta$), the calculations are greatly simplified bny the diagonal structure of our metric, so that the only non-zero symbols are the following:\\
\begin{itemize}
	\item $$\Gamma_{tr}^t={1\over 2}g^{t\alpha}(g_{t\alpha,r}+g_{r\alpha,t}-g_{tr,\alpha})={1\over 2}g^{tt}(g_{tt,r}+g_{rr,t}-0)=\partial_r\alpha(r)$$
	\item $$\Gamma_{tt}^r={1\over 2}g^{r\alpha}(g_{t\alpha,t}+g_{t\alpha,t}-g_{tt,\alpha})={1\over 2}g^{rr}(0+0-g_{tt,r})=e^{2(\alpha(r)-\beta(r))}\partial_r\alpha(r)$$
	\item $$\Gamma_{rr}^r={1\over 2}g^{r\alpha}(g_{r\alpha,r}+g_{r\alpha,r}-g_{rr,\alpha})={1\over 2}g^{rr}(g_{rr,r}+g_{rr,r}-g_{rr,r})=\partial_r\beta(r)$$
	\item $$\Gamma_{r\theta}^\theta={1\over 2}g^{\theta\alpha}(g_{r\alpha,\theta}+g_{\theta\alpha,r}-g_{r\theta,\alpha})={1\over 2}g^{\theta\theta}(g_{r\theta,\theta}+g_{\theta\theta,r}-g_{r\theta,\theta})={1\over r}$$
	\item $$\Gamma_{\theta\theta}^r={1\over 2}g^{r\alpha}(g_{\theta\alpha,\theta}+g_{\theta\alpha,\theta}-g_{\theta\theta,\alpha})={1\over 2}g^{rr}(g_{\theta r,\theta}+g_{\theta r,\theta}-g_{\theta\theta,r})=-e^{-2\beta(r)}r$$
	\item $$\Gamma_{r\phi}^\phi={1\over 2}g^{\phi\alpha}(g_{r\alpha,\phi}+g_{\phi\alpha,r}-g_{r\phi,\alpha})={1\over 2}g^{\phi\phi}(g_{r\phi,\phi}+g_{\phi\phi,r}-g_{r\phi,\phi})={1\over r}$$
	\item $$\Gamma_{\phi\phi}^r={1\over 2}g^{r\alpha}(g_{\phi\alpha,\phi}+g_{\phi\alpha,\phi}-g_{\phi\phi,\alpha})={1\over 2}g^{rr}(g_{\phi r,\phi}+g_{\phi r,\phi}-g_{\phi\phi,r})=-re^{-2\beta(r)}sin^2(\theta)$$
	\item $$\Gamma_{\phi\phi}^\theta={1\over 2}g^{\theta\alpha}(g_{\phi\alpha,\phi}+g_{\phi\alpha,\phi}-g_{\phi\phi,\alpha})={1\over 2}g^{\theta\theta}(g_{\phi\theta,\phi}+g_{\phi\theta,\phi}-g_{\phi\phi,\theta})=-cos(\theta)sin(\theta)$$
	\item $$\Gamma_{\theta\phi}^\phi={1\over 2}g^{\phi\alpha}(g_{\theta\alpha,\phi}+g_{\phi\alpha,\theta}-g_{\theta\phi,\alpha})={1\over 2}g^{\phi\alpha}(g_{\theta\phi,\phi}+g_{\phi\phi,\theta}-g_{\theta\phi,\phi})=ctan(\theta)$$
\end{itemize}
Now we can finally calculate the Riemann tensor by making use to the known relation 
$$R^\rho_{\sigma\mu\nu}=\partial_\mu\Gamma^\rho_{\nu\sigma}-\partial_\nu\Gamma^\rho_{\mu\sigma}+\Gamma^\rho_{\mu\lambda}\Gamma^\lambda_{\nu\sigma}-\Gamma^\rho_{\nu\lambda}\Gamma^\lambda_{\mu\sigma}$$ 
The Riemann tensor satiefies the folloqing properties:
\begin{itemize}
	\item $$R_{ijkl}=R_{klij}$$
	\item $$R_{ijkl}=-R_{ijlk}=-R_{jikl}$$
	\item $$R_{ijkl}+R_{iklj}+R_{iljk}=0$$
	\item $$R_{ijkl;m}+R_{ijm;kl}+R_{ijlm;k}=0$$
\end{itemize}
It is known that due to its properties this quantity has only 20 independent components. The non-vanishing components are the following:
\begin{itemize}
	\item 
\end{itemize}
Finally, the Ricci scalar takes thew following form:
$$R=-2e^{-2\beta}\bigg[\partial_r^2\alpha+(\partial_r\alpha)+{2\over r}(\partial_r\alpha-\partial_r\beta)+{1\over r^2}(1-e^{2\beta})\bigg]$$
All of the components of this tensor must vanish independently. Thus, we can set: $R_{tt}+R_{rr}=0$ which gives us:
$${2\over r}\partial_r (\alpha-\beta)=0$$ 
from which we get that $\alpha=-\beta$. Now setting $R_{\theta\theta}=0$ we obtain:
$$e^2\alpha(2r\partial_r\alpha+1)=1$$
This condition can be re-written by allplying the Leibniz rule in the opposite way.
$$\partial_r(re^{2\alpha})=1\longrightarrow e^{2\alpha}=1-{C/r}$$
where $C$ is a constant to determine. The calculations have come to an end: the only thing which is left to do is to find the correct value for the constant. With the previous knowledge, the metric takes the following form:
$$ds^2=-\bigg(1-{C\over r}\bigg)dt^2+\bigg(1-{C\over r}\bigg)^{-1}dr^2+r^2d\Omega^2$$
To correctly establish the value of $C$ we can look at the weak field limit. In general, it is known that the component $g_{tt}$ takes the form:
$$g_{tt}=-1\bigg(1-2V\bigg)=-(1-{2GM\over r})$$
In order to let our theory be compatible with the Newtonian limit we simply set $C=2GM$. This value will be of great importance in the description of the physical properties of a Black hole.
\section{Singularities}
In the previuos section we were able to fully charactierze the Schwarzschild metric, which takes the general form:
$$ds^2=-\bigg(1-{2GM\over r}\bigg)dt^2+\bigg(1-{2GM\over r}\bigg)^{-1}dr^2+r^2d\Omega^2$$
However, we see that we have some problems for certain specific values of the coordinates. In particular, when $r=0$ and $r=2GM$ the first two components of the metric diverge. However, we must recall that in general the "bad" behaviour of the metric might be related, in some points, to the choice of the coordinates. To better understand this, one should remember that the metric in a plane in spherical coordinates is ill defined for $r=0$, even though the point on the manifoild is regular. To understand if the problems that arise in our metric are related to the coordinates that we have used, or if they derive from the structure of the manifold, we can look at scalars. In particular, we check if in our problematic points the curvature becomes infinite. The curvature is measured through the Riemann tensor. However, we would like to study some scalars, which do not depend on the change of coordinates. The set of scalars which we can construct from the Riemann tensor is the following:
$$R=g^{jk}R^i_{jik};\hspace{5pt}R_{ij}R^{ij};\hspace{5pt}R_{ijkl}R^{ijkl};\hspace{5pt}R_{ijkl}^{klhf}R_{hf}^{ij}$$ 
If any of these scalars becomes infinite when evaluated in a certain point, we call that point singularity.
$r=0$ is indeed a singularity since one can show that
$$R_{ijkl}R^{ijkl}\propto {1\over r^6}$$
The truly problematic point is $r=2GM$ since one could see that all of the previous scalars do not diverge if computed in $r=2GM$.\\
\\
Before continuing, it is worth noticing that our analysis supposes that the quantity $r=2GM$ is well outside the interior of the source. This is generally true for black holes, but does not hold for stars in general. In the case of the sun, this point is well below the surface. In those cases, our equation is simply wrong since we are not in vacuum anymore.
\section{Geodesics}
To calculate geodesics one could simply solve the geodesic equations $${d^2x^\mu\over d\lambda^2}=-\Gamma_{\alpha\beta}^\mu{dx^\alpha\over d\lambda}{dx^\beta\over d\lambda}$$
which however result to be very complicated. Instead, one can make use of the many assumptions we made on our system. In particular, we supposed that our source was static and spherically symmetric. Those requirements translate into the existence of 4 Killing vectors: 3 for rotations and 1 for time translations. Each of those Killing vectors will lead to a conserved quantity. It is in fact true that, if $K^\mu$ is a Killing vector, then the quantity $$K^\mu u_\mu$$
is constant. We also know that along the geodesic there's an additional conserved quantity, which we will call
$$\epsilon=u_\mu u^\mu$$
Which for massive particles is -1 and for massless particles is 0.\\
\\
In analogy with classical mechanics, the presence of a symmetry under rotation implies the conservation of the modulus and the direction of the angular momentum. In terms of our differential geometry apparatus the conservation of the direction of angular momentum implies that the trajectory of a test particle will lie on a plane. Thus, we are allowed to choose this plane as out equatorial plane. This means rotating our coordinate system such that $\theta={\pi\over2}$.\\
The conservation of the modulus of angular momentum instead is due to the killing vector $L^\mu=(\partial_\phi)^\mu=(0,0,0,1)$. Lowering the index we get:
$$L_\mu=g_{\mu\nu}L^\nu=g_{\mu\phi}L^\phi=(0,0,0,r^2sin^2(\theta))=(0,0,0,r^2)$$
where we have used the fact that $\sin({\pi\over2})=1$.
Then, the conserved quantity is $$L_\mu u^\mu=r^2{d\phi\over d\lambda}=L$$
As for the time-like Killing vector, it imoplies the conservation of energy: 
$$K^\mu=(\partial_t)^\mu=(1,0,0,0)$$
once again we lower the indices:
$$K_\mu=g_{\mu\nu}K^\nu=g_{\mu t}K^t=-\bigg(1-{2GM\over r},0,0,0\bigg)$$
We call Killing energy the conserved quantity:
$$E=-K^\mu u_\mu=\bigg(1-{2GM\over r}\bigg){dt\over d\lambda}$$
Our idea now is to find subsitute the results found into the expression of $\epsilon$ to obtain the last geodesic equation.
$$\epsilon=g_{\mu\nu}u^\mu u^\nu=g_{tt}\bigg({dt\over d\lambda}\bigg)^2+g_{rr}\bigg({dr\over d\lambda}\bigg)^2+g_{\theta\theta}\bigg({d\theta\over d\lambda}\bigg)^2+g_{\phi\phi}\bigg({dt\over d\lambda}\bigg)^2$$
However ${d\theta\over d\lambda}=0$ since the evolution of the system is bound to a plane. Substituting what we previously found we obtain:
$$0=-E^2+\bigg({dr\over d\lambda}\bigg)^2+\bigg(1-{2GM\over r}\bigg)\bigg({L^2\over r^2}+\epsilon\bigg)$$
\section{Black Holes}
We begin our analysis by studying radial null trajectories. Those correspond to $ds^2=0$ and $\theta=0=\phi$ so that the metric takes the form:
$$ds^2=0=-\bigg(1-{2GM\over r}\bigg)dt^2+\bigg(1-{2GM\over r}\bigg)^{-1}dr^2$$ 
shich obviously implies that
$${dt\over dr}=\pm \bigg(1-{2GM\over r}\bigg)^{-1}$$
This result is the slope of the light cones on a space-time diagram in the $(t,r)$ coordinates. We can see that for $r\rightarrow+\infty$ the slope tend to be at $45°$ inclination, which is the same result of the flat Minkowsky space. This makes sense physically since, being infinitely far away from the source, we do not feel its gravitational pull. Instead, for $r\rightarrow 2GM^\pm$ we get ${dt\over dr}\rightarrow\pm\infty$. So it appears that in those coordinates the light cones close up when we approach the surface $r=2GM$. This means that a photon that approaches this very suirface never seems to actually get there (not in a finite time). In our next analysis, through changes of coordiantes, we will show that this inability to reach the horizon $2GM$ is only apparent. However, an observer infinitely far away would never be able to tell.
\subsection{The Tortoise coordinates}
To better undesrstand if a photon can reach the horizon in a finite amount of proper time, we start by making the following substitution, called Tortoise coordinate:
$$r^*=r+2GMln\bigg({r\over 2GM}-1\bigg)$$
Note that this coordinate is ill defined for $r\leq2GM$ since the logarithm becomes complex after this limit.
Obviously the differential is:
$$dr^*=dr\bigg(1+\bigg({r\over 2GM}-1\bigg)^{-1}\bigg)=\bigg(1-{2GM\over r}\bigg)^{-1}dr$$
Substituting this into the metric we get:
$$ds^2=\bigg(1-{2GM\over r}\bigg)(-dt^2+{dr^*}^2)+r^2d\Omega$$
Now when evaluating null trajectories we find that the light cones do not close up anymore:
$${dt\over {dr^*}}=\pm1$$
However, we have basically pushed away the surface $r=2GM$. In fact we recover this surface only at $r^*\rightarrow -\infty$.
\subsection{The Eddington-Finkelstein coordinates}
We define some new coordinates which are naturally adapted to the null geodesics. In particular, define
$$v=t+r^*;\hspace{5pt} u=t-r^*$$
Obviously the differentials are:
$$dv=dt+dr^*;\hspace{5pt} du=dt-dr^*$$
We see that ${dv\over dr^*}={dt\over dr^*}+1$ and ${du\over dr^*}={dt\over dr^*}-1$ so that for outgoing null trajectories ${du\over dr^*}=0$ and for ingoing null trajectories ${dv\over dr^*}=0$. In this sense those coordinates are adapted to the null geodesics.\\
Let un substitute one at a time those new variables and see how do they affect the metric: by substituting $v$ we get:
$$ds^2=\bigg(1-{2GM\over r}\bigg)(-dv^2+2dvdr^*)+r^2d\Omega$$
And substituting $r^*$ as a function of $r$ we find:
$$ds^2=-\bigg(1-{2GM\over r}\bigg)dv^2+2dvdr+r^2d\Omega$$
With this substitution we see that the only apparent feature of the metric is the vanishing of the time coefficient at the horizon $r=2GM$. However, everyrhing is regular at this surface.\\
\\
Let us now study radial null geodesics: 
$$0=-\bigg(1-{2GM\over r}\bigg)dv^2+2dvdr$$
this implies the following result:
\\
\begin{center}
	\begin{math}
		\displaystyle {dv\over dr}=
		\begin{cases}
			0 \mbox{\hspace{68pt}Ingoing}\\
			2\bigg(1-{2GM\over r}\bigg)^{-1} \mbox{  Outgoing}
		\end{cases}
	\end{math}
\end{center}
Thus, the light cones have one axis which is alwasy parallel to the direction $v=constant$ with $r$ and the other which varies in relation to the horizon. Before looking at the actual precise behaviuor of the light cones, we see that there is no bad behaviour of the coordinates at the horizon. The apparent singularity was only suggested by the Schwarzschild coordinates.\\
As for the light cones, we can see from Figure 1, they appear to tilt as we approach the singularity. In particular, at the horizon the outgoing light trajectory is perfectly vertical, while past it at $r<2GM$ it's even more tilted. This not only means that once passed the horizon, it is impossible for an object to exit it. Further more, every outgoing trajectory still points towards the singularity!\\
\\
\begin{center}
	\begin{figure}[h!]
		\centering
		\includegraphics{RelGenv.png}
		\caption{Light cones in the Schwarzschild coordinates with Eddington-Finkelstein coordinate v.}
	\end{figure}
\end{center}
We can easily repaeat the previous calculations for the other variable $u$. Everything is still the same, except for a minus sign in front of the factor $r^*$. Thus we find the following metric:
$$ds^2=-\bigg(1-{2GM\over r}\bigg)du^2-2dudr+r^2d\Omega$$
Now, looking again at null trajectories we get:
$$0=-\bigg(1-{2GM\over r}\bigg)du^2-2dudr$$
which in turn implies that:
\begin{center}
	\begin{math}
		\displaystyle {du\over dr}=
		\begin{cases}
			0 \mbox{\hspace{68pt}Outgoing}\\
			-2\bigg(1-{2GM\over r}\bigg)^{-1} \mbox{  Ingoing}
		\end{cases}
	\end{math}
\end{center}
Let us look at Figure 2. This time the light cones have the outgoing axis always pointing on the direction of $u$ constant. The ingoing axis instead tilts in relation to the horizon and at $r=2GM$ is perfectly parallel to the axis $u$. Past this surface instead, we see that this axis tilts even more. This means that every future directed trajectory is in the direction of increasing $r$.
\begin{center}
	\begin{figure}[h!]
		\centering
		\includegraphics{RelGenu.png}
		\caption{Light cones in the Schwarzschild coordinates with Eddington-Finkelstein coordinate u.}
	\end{figure}
\end{center}
We then see that the first substitution describes a black hole since once we get past the horizon we are not able to go back. Furthermore, we are bound to move towards the singularity. The second substitution describes a white hole, since once passed (in an outgoing way) the horizon we cannot enter it again. To this we must add that when we are inside it we can only move away from the singularity.\\
\\
\end{document}
